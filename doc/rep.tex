        % % Title and author(s)
%%%%%%%%%%%%%%%%%%%%%%%%%%%%%%%%%%%%%%%%%%%%%%%%%%%%%%%
\title{Topics in Database Management Systems Project Report: Skipping Filters is Better}
\author{Nicholas Corrado \and Xiating Ouyang}
\date{}
%%%%%%%%%%%%%%%%%%%%%%%%%%%%%%%%%%%%%%%%%%%%%%%%%%%%%%%
\documentclass[11pt]{article}
%%%%%%%%%%%%%%%%%%%%%%%%%%%%%%%%%%%%%%%%%%%%%%%%%%%%%%%
% %
% % The next command allows your in import encapsulated
% % postscript files, .epsf or .eps files, which
% % contain vector graphic image data.
% %
%%%%%%%%%%%%%%%%%%%%%%%%%%%%%%%%%%%%%%%%%%%%%%%%%%%%%%%
\usepackage{graphicx}
%\usepackage{charter,eulervm}

%\renewcommand{\baselinestretch}{1.5}
\setcounter{secnumdepth}{3} % default value for 'report' class is "2"
\usepackage{amsthm,amsmath,amssymb,upgreek,marvosym,mathtools}
\usepackage{array}
\usepackage{makeidx}  % allows for indexgeneration
\usepackage{paralist}
\usepackage{subfig}
\usepackage{tabularx}
\usepackage{tabu}
\usepackage{comment}
\usepackage[nottoc]{tocbibind}
\usepackage[usenames,dvipsnames]{color}
\usepackage[pdftex,breaklinks,colorlinks,citecolor={blue}, linkcolor={blue},urlcolor=Maroon]{hyperref}
\usepackage{tkz-graph}
\usepackage{geometry}
 \geometry{
 a4paper,
 total={170mm,257mm},
 left=25mm,
 top=20mm,
 right=25mm,
 }
\usetikzlibrary{automata, positioning,arrows,shapes,decorations.pathmorphing}

 \tikzset{
->, % makes the edges directed
>=stealth, % makes the arrow heads bold
node distance=3cm, % specifies the minimum distance between two nodes. Change if necessary.
every state/.style={thick, fill=gray!10}, % sets the properties for each ’state’ node
initial text=$ $, % sets the text that appears on the start arrow
}

\newtheorem{theorem}{Theorem}[section]
\newtheorem{lemma}{Lemma}[section]
\newtheorem{reduction}{Reduction}[section]
\newtheorem{proposition}{Proposition}[section]
\newtheorem{scolium}{Scolium}[section]   %% And a not so common one.
\newtheorem{definition}{Definition}[section]
\newtheorem{conjecture}{Conjecture}[section]
\newtheorem{corollary}{Corollary}[section]
%\newenvironment{proof}{{\sc Proof:}}{~\hfill QED}
\newenvironment{AMS}{}{}
\newenvironment{keywords}{}{}
\DeclarePairedDelimiter{\norm}{\lVert}{\rVert}
\newcommand{\todo}{(TO BE CONTINUED...)}

\newcommand{\paris}[1]{{\color{blue} Paris: [{#1}]}}


\newcommand{\trans}[1]{
	#1^\mathsf{T}
}

\newcommand{\db}{$\mathbf{db}$}
\newcommand{\sjfq}{\texttt{sjfCQA}}
\newcommand{\bcq}{\texttt{bcq}}
\newcommand{\problem}[1]{\textsc{certainty}($#1$)}
\newcommand{\FO}{$\mathbf{FO}$}
\newcommand{\PTIME}{$\mathbf{P}$}
\newcommand{\LSPACE}{$\mathbf{L}$}
\newcommand{\coNP}{$\mathbf{coNP}$}
\newcommand{\und}[1]{\underline{#1}}
\newcommand{\NL}{$\mathbf{NL}$}
\newcommand{\JOIN}{\bowtie}

\begin{document}
\newpage
\maketitle


\abstract{}


\section{Introduction}

Performing join operations on database management systems based on Star Schema is a fundamental and task in industry. \textit{Lookahead Information Passing (LIP)} is a technique that constructs a set of Bloom Filters on the dimension tables and adaptively applies them in certain sequence on the fact table, provably as efficient and robust as computing the join according to the optimal joining sequence \cite{zhu2017looking}. However, the cost is the huge amount of memory and cache used to store the set of LIP filters: Replacing the LIP filters in cache for a new LIP filter causes significant delay, and if the new LIP filter is not selective skipping it might improve the performance.

This project aims at investigating the effect of skipping certain LIP filter on improving the performance of LIP, and if possible, derive a theoretical guarantee on the performance of LIP against the optimal joining sequence. 



\section{Lookahead Information Passing}

Let $F$ be the fact table and $D_i$ the dimension tables for $1 \leq i \leq n$. We denote the number of facts in $F$ and each $D_i$ as $|F|$ and $|D_i|$. A LIP filter on $D_i$ is implemented using a Bloom filter, with false positive rate $\varepsilon$. The true selectivity $\sigma_i$ of $D_i$ on fact table $F$ is given by $\sigma_i = |D_i \JOIN_{k_i} F| / |F|,$ where $k_i$ is the foreign key of $D_i$ in $F$. The \texttt{LIP-join} algorithm, depicted in Figure \ref{fig:lip}, computes the indices of tuples in $F$ that pass the filtering of each Bloom filter of $D_i$.

The partition in \cite{zhu2017looking} satisfies that $|F_{t+1}| = 2|F_{t}|$ at line 5, and the algorithm approximates the true selectiveness $\sigma_i$ of each dimension $D_i$ using $pass[i]/count[i]$, the aggregated selectiveness since the beginning.

\begin{figure}[h!]
	\centering
	\tikz\path (0,0) node[draw, text width=.8\textwidth, rectangle, inner xsep=20pt, inner ysep=10pt]{
		\begin{minipage}[t!]{\textwidth}
			{\sc Procedure}: \texttt{LIP-join}
			\\
			{\sc Input}: a fact table $F$ and a set of $n$ Bloom filters $f_i$ for each $D_i$ with $1 \leq i \leq n$
 			\\
			{\sc Output}: Indices of tuples in $F$ that pass the filtering
			\begin{tabbing}
				Aaa\=aaA\=Aaa\=Aaa\=Aaa\=AAAAAAAAAAAAAAAAAAAAAAAAA\=A \kill
				1.\> Initialize $I = \emptyset$
				\\
				2.\> {\bf foreach } filter $f$ {\bf do}
				\\
				3.\>\> $count[f] \leftarrow 0$
				\\
				4.\>\> $pass[f] \leftarrow 0$ 
				\\
				5.\> Partition $F = \bigcup_{1 \leq t \leq T}F_t$. 
				\\
				6.\> {\bf foreach } fact block $F_t$ {\bf do} 
				\\
				7.\>\> {\bf foreach } filter $f$ in order {\bf do}
				\\
				8.\>\>\> {\bf foreach} index $j \in F_t$ {\bf do}
				\\
				9.\>\>\>\> $count[f] \leftarrow count[f] + 1$
				\\
				10.\>\>\>\> {\bf if }$f$ contains $F_t[j]$ 
				\\
				11.\>\>\>\>\> $I \leftarrow I \cup \{j\}$ 
				\\
				12.\>\>\>\>\> $pass[f] \leftarrow pass[f] + 1$
				\\
				13.\>\> {\bf sort} filters $f$ in nondesending order of $pass[f]/count[f]$
				\\
				14.\> {\bf return } $I$
			\end{tabbing}  
		\end{minipage}
	};
	\caption{The LIP algorithm for computing the joins.}
	\label{fig:lip}
\end{figure}


\subsection{Discussion}



\section{Skipping LIP Filters}

In this section, we present our modified \texttt{LIP-join-skip} algorithm and discuss some implementation details.

\subsection{Implementation}

We implemented the traditional hash-join algorithm, the \texttt{LIP-join} algorithm and our modified \texttt{LIP-joip-skip} on top of apache arrow in C++. The code is available at \url{https://github.com/NicholasCorrado/CS764}.


\section{Experiments}


\section{Concluding remarks}




\bibliography{rep}{}
\bibliographystyle{plain}

\end{document}
